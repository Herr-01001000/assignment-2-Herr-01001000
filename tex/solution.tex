\documentclass[11pt,a4paper,leqno]{article}
\usepackage{a4wide}
\usepackage[T1]{fontenc}
\usepackage[utf8]{inputenc}
\usepackage{float, afterpage, rotating, graphicx}
\usepackage{longtable, booktabs, tabularx}
\usepackage{verbatim}
\usepackage{eurosym, calc, chngcntr}
\usepackage{amsmath, amssymb, amsfonts, amsthm, bm, delarray, mathtools} 
\usepackage{caption}
\usepackage{tkz-graph}
\usetikzlibrary{arrows,positioning,snakes,shapes,shapes.multipart,patterns,mindmap,shadows}

 \usepackage[backend=biber, natbib=true, bibencoding=inputenc, bibstyle=authoryear-ibid, citestyle=authoryear-comp, maxnames=10]{biblatex}
 %\bibliography{bib/hmg}

\usepackage[unicode=true]{hyperref}
\hypersetup{colorlinks=true, linkcolor=black, anchorcolor=black, citecolor=black, filecolor=black, menucolor=black, runcolor=black, urlcolor=black}
\setlength{\parskip}{.5ex}
\setlength{\parindent}{0ex}

\theoremstyle{definition}
\newtheorem{exercise}{Exercise}
\renewcommand{\theenumi}{\roman{enumi}}

% Set this counter to "first exercise of the week minus one".
\setcounter{exercise}{0}

\begin{document}

\begin{center}
    \begin{large}
        \textbf{
        Effective programming practices for economists\\
        Universität Bonn, Winter 2018/19 \\[2ex]
        Exercise solution\\[2ex]
%        Group XXXX\\[2ex]
        Wenxin Hu
        }
    \end{large}
\end{center}


\begin{exercise}
    Task 4: Write the Cobb-Douglas and CES production functions that accept arbitrary numbers of inputs.
    \begin{enumerate}
        \item The Cobb-Douglas production function that accept arbitrary numbers of inputs should be like this:
        $$y = a \cdot \prod_{j=1}^J x_j^{\gamma_j}$$
        \item The Constant Elasticity of Subsititution (CES) production function is given by:
        $$y = a \cdot \bigg(\sum_{j=1}^J \gamma_j x_j^{-\rho} \bigg)^{-\frac{1}{\rho}}$$
        \item The robust general CES production function is like this:
        \[
            y = \begin{dcases}
            a \cdot \bigg(\sum_{j=1}^J \gamma_j x_j^{-\rho} \bigg)^{-\frac{1}{\rho}} & , \rho \neq 0 \\
            a \cdot \prod_{j=1}^J x_j^{\gamma_j} & , \rho = 0 \\
            \end{dcases}
        \]
        \item Prove when $\rho$ approaches zero, the output of the CES function is close to the output of the Cobb-Douglas function:

        First, we put logarithm on both sides of the Cobb-Douglas function.
        $$ \ln y = \ln a + \sum_{j=1}^J \gamma_j \ln x_j$$

        Similarly, we put logarithm on both sides of the CES function and take the limit when $\rho$ approaches 0.
        $$\ln y = \ln a - \frac{\ln (\sum_{j=1}^J\gamma_jx_j^{-\rho})}{\rho}$$

        \begin{align*}
            \lim_{\rho \rightarrow 0} \ln y &= \ln a - \lim_{\rho \rightarrow 0} \frac{\ln(\sum_{j=1}^J\gamma_jx_j^{-\rho})}{\rho}\\ 
            &= \ln a - \lim_{\rho \rightarrow 0} \frac{\partial \ln(\sum_{j=1}^J\gamma_jx_j^{-\rho})}{\partial \rho} \\
            &= \ln a + \lim_{\rho \rightarrow 0} \frac{\sum_{j=1}^J\gamma_jx_j^{-\rho} \ln x_j}{\sum_{j=1}^J\gamma_jx_j^{-\rho}}\\
            &= \ln a + \sum_{j=1}^J \gamma_j \ln x_j
        \end{align*}

        Then we can see that the limit output of the CES function is the limit output of the Cobb-Douglas function when the limit of $\rho \rightarrow 0$.

    \end{enumerate}
\end{exercise}


\begin{exercise}
    The advantages of splitting up the code.  
    \begin{enumerate}
        \item 
        \item      
    \end{enumerate}
\end{exercise}


% \printbibliography 

\end{document}